\documentclass[10pt, a4paper, twoside, openright]{article}
% \documentclass{book}
\usepackage[english, italian]{babel}
% \usepackage[italian, english]{babel}
\usepackage[utf8]{inputenc} % needed for bibtex
\usepackage{mathrsfs}
\usepackage{amsthm}
\usepackage{amsmath}
\usepackage{amssymb}
\usepackage{mathtools}
%\usepackage{nccmath} % mfrac
%\mathbb needs amsfonts or amssymb
\usepackage{amsfonts}
\usepackage{bm} % bold symbols math

% \usepackage[overload]{empheq} % left\{ for align % no crash kile ?
\usepackage{cases}
\usepackage[usenames,dvipsnames]{xcolor} % before tikz, options for more colors
\definecolor{light-gray}{gray}{0.95}

\usepackage{booktabs}
\usepackage{caption}
\usepackage{subfig}
% \captionsetup[figure]{width=.85\textwidth}
\captionsetup[subfigure]{margin=0.5cm}
\usepackage{tikz}
\usetikzlibrary{matrix}
\usetikzlibrary{positioning}
% set arrows as stealth fighter jets
\tikzset{>=stealth}
% bezier
\usetikzlibrary{decorations.pathreplacing}
\tikzset{%
  show curve controls/.style={
    postaction={
      decoration={
        show path construction,
        curveto code={
          \draw [blue] 
            (\tikzinputsegmentfirst) -- (\tikzinputsegmentsupporta)
            (\tikzinputsegmentlast) -- (\tikzinputsegmentsupportb);
          \fill [red, opacity=0.5] 
            (\tikzinputsegmentsupporta) circle [radius=.5ex]
            (\tikzinputsegmentsupportb) circle [radius=.5ex];
        }
      },
      decorate
}}}
\tikzstyle{mybox} = [draw=gray, fill=light-gray, very thick,
    rectangle, rounded corners, inner sep=10pt, inner ysep=20pt]
\tikzstyle{mytitle} =[fill=gray, text=white]

% plots
\usepackage{pgfplots}

%\usepackage{color}

\usepackage{xcolor}
\definecolor{bookColor}{cmyk}{1	, 1  , 0   , 0}  % 0.90\% of black
%\color{bookColor}

\usepackage{hyperref}
\hypersetup{pdftex,colorlinks=true,allcolors=blue}
\usepackage{hypcap}
\usepackage{etoolbox}

\usepackage{csquotes}
%\usepackage[autostyle, italian=guillemets]{csquotes}

% \usepackage[chapter]{placeins} % floatbarrier

\usepackage[backend=biber, style=alphabetic]{biblatex}
\addbibresource{sources.bib}

\numberwithin{equation}{section}

\theoremstyle{definition}
\newtheorem{definition}[subsection]{Definition}
\theoremstyle{plain}
\newtheorem{theorem}[subsection]{Theorem}
\theoremstyle{plain}
\newtheorem{corollary}[subsection]{Corollary}
\theoremstyle{plain}
\newtheorem{proposition}[subsection]{Proposition}
\theoremstyle{plain}
\newtheorem{remark}[subsection]{Remark}
\theoremstyle{plain}
\newtheorem{lemma}[subsection]{Lemma}
\theoremstyle{plain}
\newtheorem{example}[subsection]{Example}

\theoremstyle{plain}
\newtheorem{assumption}[subsection]{Assumption}
\theoremstyle{plain}
\newtheorem{problem}[subsection]{Problem}

\DeclareMathOperator{\divergence}{div}
\DeclareMathOperator{\curl}{curl}
\DeclareMathOperator{\real}{Re}
\DeclareMathOperator{\imag}{Im}

\providetoggle{verbose}
\settoggle{verbose}{true}
\providetoggle{old}
\settoggle{old}{false}
\providetoggle{fig}
\settoggle{fig}{true}
% \toggletrue{paper}
% \togglefalse{paper}
\renewcommand{\i}{\textup{i}}
\let\phi\varphi
\let\epsilon\varepsilon

\newcommand{\upin}{\textup{ in }}
\newcommand{\upon}{\textup{ on }}
%%%%%%%%%%%%%%%%%%%%%%%%%%%%%%
\usepackage{amssymb,tikz}

\newcommand{\mysetminusD}{\hbox{\tikz{\draw[line width=0.6pt,line cap=round] (3pt,0) -- (0,6pt);}}}
\newcommand{\mysetminusT}{\mysetminusD}
\newcommand{\mysetminusS}{\hbox{\tikz{\draw[line width=0.45pt,line cap=round] (2pt,0) -- (0,4pt);}}}
\newcommand{\mysetminusSS}{\hbox{\tikz{\draw[line width=0.4pt,line cap=round] (1.5pt,0) -- (0,3pt);}}}

\newcommand{\mysetminus}{\mathbin{\mathchoice{\mysetminusD}{\mysetminusT}{\mysetminusS}{\mysetminusSS}}}
%%%%%%%%%%%%%%%%%%%%%%%%%%%%%%%%%%


\usepackage{environ}
\NewEnviron{mybox}{%
\begin{center}
\colorbox{light-gray}{\color{black}\parbox{\textwidth}{%
% \fcolorbox{gray}{light-gray}{
\BODY
}}
\end{center}
}

\usepackage{fancyhdr}
\newcommand{\fncyblank}{\fancyhf{}}
%%%%%%%%%%%%%%%%%%%%%%%%%%%%%%%%%%%%%%%%%%%% abstract already defined
% \newenvironment{abstract} %
% {\cleardoublepage\fncyblank\null\vfill\begin{center} %
% \bfseries\abstractname\end{center}} %
% {\vfill\null}
%%%%%%%%%%%%%%%%%%%%%%%%%%%%%%%%%%%%%%%%%%%%%%%%%%%%%%%%%%%%%%%%%%%%%%%%%
\title{The Boundary Integral Method}
\author{Giacomo Milan}
% \date{3 Ottobre 2017} % oppure anno accademico

\begin{document}
\maketitle
\tableofcontents
% \listoffigures
%\listoftables

\section{Boundary Integral Equations BIE}
In this section we introduce the theory of Boundary Integral Equations BIE considering as boundary 
value problem BVP the Laplace equation in an open set $\Omega$ of class $C^2$
\begin{eqnarray}
 \Delta u = 0, & \textup{ in }\Omega,\label{eq:laplace-in}\\
 u = g, & \textup{ on }\partial\Omega.\label{eq:laplace-on}
\end{eqnarray}
The aim is to represent the solution $u$ through \emph{representation formulas} 
in terms of its boundary data $u|_{\partial \Omega}$, $\partial_\nu u|_{\partial \Omega}$, and 
of $\Delta u$. 
We generalize the previous example to the problem $Lu = f$ in $\mathcal{D}'(\Omega)$, in the 
distributional sense, denoting by $L$ a differential operator with smooth coefficients.
Let $\Phi(z,z_0)$ denote the \emph{fundamental solution} which solves
\begin{equation}
\label{eq:fundamental-solution}
 L\Phi=\delta_{z_0},\quad\upin\mathcal{D}'(\mathbb{R}^m).
\end{equation}
For the Laplace equation with $L=-\Delta$, it can be explicitly expressed as
\begin{equation}
\label{eq:definition-Phi-23}
  \Phi(x,x_0)=
  \left\{
  \begin{aligned}
   &\dfrac{1}{2\pi}\log\dfrac{1}{| x - x_0|}, && m=2, \\
   &\dfrac{1}{4\pi}\dfrac{1}{| x  - x_0|}, && m=3.
  \end{aligned}
  \right.
\end{equation}
Let $\Omega$ be a domain of class $C^2$ (or even Lipschitz regular),
and $u \in C^2(\Omega)\cap C^1(\overline{\Omega})$,
than, by classical Green's formula,
there holds the following integral representation formula
\begin{equation}
  \label{eq:representation-formula}
  - u(z_0) = \int_\Omega\Delta u\,\Phi_{z_0}\,dz 
  + \int_{\partial \Omega}\big(u\, \partial_{\nu(z)} \Phi_{z_0}
  - \partial_\nu u\,\Phi_{z_0}\big)\,dz, \quad z_0 \in \Omega.
\end{equation}
% \begin{proposition}[Classical Green's Representation]
%  Let $D$ be a Lipschitz domain and $u,v \in C^2(D)\cap C^1(\overline{D})$ be regular functions, then integration by parts can be expressed as
%   \begin{equation}
%   \int_D (u \, \Delta v - \Delta u \, v)\, dx = \int_{\partial D}(u \, \partial_\nu v - \partial_\nu u \, v)\,d\sigma = \mathcal{R}_{\partial D}(u,v).
%   \end{equation}
% \end{proposition}
% \begin{proposition}[Representation of harmonics]
%  Indeed formally, for distributions in $\mathbb{R}^m$ which take to account the boundary of $D$, $\langle u,\Delta_p\Phi(p,p_0)\rangle = \langle u,-\delta_{p_0}\rangle = - u(p_0)$.
% \end{proposition}
In our case, any solution $u$ of \eqref{eq:laplace-in} can be represented as
\begin{equation}
  \label{eq:representation-formula-harmonics}
  - u(z_0) = \int_{\partial \Omega}\big(u\, \partial_{\nu(z)} \Phi_{z_0}
  - \partial_\nu u\,\Phi_{z_0}\big)\,dz, \quad z_0 \in \Omega,
\end{equation}
as the sum of two layer potentials defined on $\partial\Omega$.
Given a density $\psi\in C(\partial \Omega)$, we define
\begin{enumerate}
  \item the \emph{single layer} potential
   \begin{equation}
    \mathcal{S}(\partial \Omega,\psi)(x)\coloneqq \int_{\partial \Omega} \Phi(x, y)\psi(y)\, dy,\quad x\in\mathbb{R}^m \backslash\partial \Omega, \label{eq:definition-single-layer}
   \end{equation}
  \item the \emph{double layer} potential
   \begin{equation}
    \mathcal{D}(\partial \Omega,\psi)(x)\coloneqq \int_{\partial \Omega} \partial_{\nu(y)}\Phi(x, y)\psi(y)\, dy,\quad x\in\mathbb{R}^m \backslash\partial \Omega. \label{eq:definition-double-layer}
   \end{equation}
\end{enumerate}
For a domain $\Omega$ of class $C^2$, there hold the well known \emph{jump relation}
which can be written as
\begin{equation}
\label{eq:jump-relations}
 [\partial_\nu\mathcal{S}]_{\partial \Omega} = -\psi,
 \quad
 [\mathcal{D}]_{\partial \Omega} = \psi.
\end{equation}
These relations can be written more detailed through integral operators 
$S, K, K': C(\partial \Omega)\to C(\partial \Omega)$, with the same expression
of the previous layer potentials (see Theorem \ref{theo:jump-relations} in the Appendix). 
We remark that they are well defined for a regular 
boundary $\partial\Omega$, thanks to the non trivial convergence of the integral of 
the singularity contained in the fundamental solution $\Phi$.
\par
The derivation of representation formulas can be done more rigorously in the 
distributional framework, introducing the notion of pseudo-differential operators.
\par
The solution of problem \eqref{eq:laplace-in}--\eqref{eq:laplace-on}, with Dirichlet data
$g$, through representation \eqref{eq:representation-formula-harmonics} containing 
the unknown $\partial_\nu u|_{\partial\Omega}$, can proceed in two ways
\begin{enumerate}
 \item eliminating the dependence of $\partial_\nu u$, constructing a representation
 formula with the \emph{Green function} for the homogeneous Neumann problem in $\Omega$, 
 that is
  \begin{alignat}{2}
  -\Delta G &= \delta_{z_0}, && \textup{ in }\Omega,\label{eq:green-neum-in}\\
  \partial_\nu G &= 0, && \textup{ on }\partial\Omega;\label{eq:green-neum-on}
  \end{alignat}
  \item finding an equation for $\partial_\nu u$. 
\end{enumerate}
We will follow the second approach, taking advantage of jump relations previously stated,
to derive a \emph{boundary integral equation}.
% We briefly present the main notions of potential theory, which we'll use in the sequel.
\appendix
\section{Potential Theory}
% \begin{definition}
%  \label{def:layer-potentials}
%  We define the following integral functions, for $\partial D$ of class $C^2$ and a 
% \end{definition}
\begin{definition}
 We denote by $S$, $K$  and $K'$ the following integral operators defined on $C(\partial D)$, where $\partial D$ is of class $C^2$
 \begin{enumerate}
  \item  $ S: C(\partial D) \to C(\partial D)$
  \begin{equation}
  S\psi(x)\coloneqq\int_{\partial D}\psi(y) \Phi(x,y)\,dy,\label{def:operator-S}
  \end{equation}
  \item  $ K,K':C(\partial D) \to C(\partial D)$
  \begin{align}
  & K\psi(x)\coloneqq\int_{\partial D} \psi(y) \partial_{\nu(y)} \Phi(x, y) \,dy =\int_{\partial D} \psi(y) \nabla_y\Phi(x, y)\cdot\nu(y) \,dy,\label{def:operator-K}\\
  & K'\psi(x)\coloneqq\int_{\partial D} \psi(y) \partial_{\nu(x)} \Phi(x, y) \,dy =\int_{\partial D} \psi(y) \nabla_x\Phi(x, y)\cdot\nu(x) \,dy.\label{def:operator-K'}
  \end{align}
 \end{enumerate}
\end{definition}
\begin{theorem}
\label{theo:jump-relations}
 Let $\psi\in C(\partial D)$, $\partial D$ be of class $C^2$, and let 
 $\mathcal{S}(\psi,\partial D)$, $\mathcal{D}(\psi,\partial D)$ be the potentials in 
 Definition \ref{def:layer-potentials}, then
 \begin{enumerate}
  \item the single layer is continuous and 
  \begin{subequations}
  \begin{align}
   \mathcal{S}^\pm(z) &\coloneqq\lim_{h\to 0^\pm}\mathcal{S}(z+h\nu(z)) = S\psi(z)=\int_{\partial D}\psi(y)\Phi(z,y)\,dy \quad z\in\partial D, \label{eq:single-pm-0}\\
   \partial_\nu\mathcal{S}^\pm(z) &\coloneqq \lim_{h\to0^\pm} \nabla\mathcal{S}(z+h\nu(z))\cdot\nu(z) =  K'\psi(z) \,\mp\,\dfrac{1}{2}\psi(z) \quad z\in\partial D,\label{eq:single-pm-1}
  \end{align}
 \end{subequations}
 \item the double layer can be continuously extended from $D$ to $\overline{D}$, from $\mathbb{R}^m\backslash \overline{D}$ to $\mathbb{R}^m\backslash D$, and
  \begin{subequations}
  \begin{align}
   \mathcal{D}^\pm(z) &= K\psi(z) \pm\dfrac{1}{2}\psi(z)\quad z\in\partial D, \label{eq:double-pm-0}\\
   \partial_\nu\mathcal{D}^+(z) &= \partial_\nu\mathcal{D}^-(z) \quad z\in\partial D. \label{eq:double-pm-1}
  \end{align}
  \end{subequations}
 \end{enumerate}
\end{theorem}


The formula \eqref{eq:representation-formula-harmonics} can be extended to $z_0 \in 
\mathbb{R}^m\backslash\partial\Omega$ and 

The rigorous derivation of the 
representation formulas 

% \include{notes/chapter-direct.tex}

%\nocite{*}
\printbibliography % biber
% \bibliographystyle{plain} % bibtex
% \bibliography{sources} % bibtex
\end{document}

\colorbox{Orchid}{

% \framebox[0.5\textwidth][l]{
 \parbox{0.5\textwidth}{
  in evidenza:
  \begin{description}
         \item[parola chiave 1]: ...sssss;
         \item[parola chiave 2]: ....
  \end{description}

 }
%  }
}


%%%%%%%%%%%%%%%%%%%%%%%%%%%%%%%%%%%%
\begin{alignat}{2}[left=\empheqlbrace]
 & u_t = H(x,t,Du) & \quad&\text{in }\mathbb{R}^n × (0,T) \\[\medskipamount]
  & u(x,0)=u_0(x) & &\text{in } \mathbb{R}^n
\end{alignat}
\vskip 1cm

\begin{subequations}
\begin{alignat}{2}[left=\empheqlbrace]
 & u_t = H(x,t,Du) &\quad & \text{in }\mathbb{R}^n × (0,T) \\[\medskipamount]
 & u(x,0)=u_0(x) & & \text{in } \mathbb{R}^n
\end{alignat}
\end{subequations}
%%%%%%%%%%%%%%%%%%%%%%%%%%%%%%%%%%%%%
\usepackage{cases}
\begin{numcases}{f(x)=}
   1 & $x\geq0$ \label{positive}
   \\
   0 & $x<0$ \label{negative}
\end{numcases}

See the second case \ref{negative} or the first \ref{positive}
% first part is ALREADY MATH MODE
\begin{subnumcases}{f(x)=}
   1 & $x\geq0$ \label{positive-subnum}
   \\
   0 & $x<0$ \label{negative-subnum}
\end{subnumcases}
%%%%%%%%%%%%%%%%%%%%%%%%%%%%%%%%%%%%%%%%
\begin{center}
\begin{tikzpicture}
\draw  plot[smooth, tension=0.7] coordinates {(-3.5,0.5) (-3,2.5) (-1,3.5) (1.5,3) (5,2.5) (5,0.5) (2.5,-2) (-3,-2) (-3.5,0.5)};
\end{tikzpicture}
\end{center}
%%%%%%%%%%%%%%%%%%%%%%%%%%%%%%%%%%%%%%%%%
\begin{tikzpicture}
% \draw [help lines] (-4, -1) grid (4, 5);
\draw [show curve controls]
  (-3, 4) .. controls ++(135:-1) and ++(135:1) .. (0, 4); 
% \draw [show curve controls] (-1, -1) 
%   .. controls ++(165:-1) and ++(270: 1) .. ( 1.5, 1)
  .. controls ++(165:-1) and ++(165:-1) .. ( 0, 1)
  .. controls ++(165: 1) and ++(90: 1) .. (-2, 1)
  .. controls ++(90:-1) and ++(165: 1) .. ( -1, -1);
\end{tikzpicture}
%%%%%%%%%%%%%%%%%%%%%%%%%%%%%%%%%%%%%%%%%%
\tikzstyle{mybox} = [draw=red, fill=blue!20, very thick,
    rectangle, rounded corners, inner sep=10pt, inner ysep=20pt]
\tikzstyle{fancytitle} =[fill=red, text=white]

\begin{tikzpicture}
\node [mybox] (box){%
    \begin{minipage}{0.50\textwidth}
        To calculate the horizontal position the kinematic differential
        equations are needed:
        \begin{align}
            \dot{n} &= u\cos\psi -v\sin\psi \\
            \dot{e} &= u\sin\psi + v\cos\psi
        \end{align}
        For small angles the following approximation can be used:
        \begin{align}
            \dot{n} &= u -v\delta_\psi \\
            \dot{e} &= u\delta_\psi + v
        \end{align}
    \end{minipage}
};
\node[fancytitle, right=10pt] at (box.north west) {A fancy title};
\node[fancytitle, rounded corners] at (box.east) {$\clubsuit$};
\end{tikzpicture}%
%%%%%%%%%%%%%%%%%%%%%%%%%%%%%%%%%%%%%%%%%%
\centering
\includegraphics[width=\textwidth]{fig/prova}
