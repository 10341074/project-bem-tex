\documentclass[10pt, a4paper]{article} % , twoside
% \documentclass{book}
\usepackage[english]{babel}
% \usepackage[italian, english]{babel}
\usepackage[utf8]{inputenc} % needed for bibtex
\usepackage{mathrsfs}
\usepackage{amsthm}
\usepackage{amsmath}
\usepackage{amssymb}
\usepackage{mathtools}
%\usepackage{nccmath} % mfracf
%\mathbb needs amsfonts or amssymb
\usepackage{amsfonts}
\usepackage{bm} % bold symbols math

% \usepackage[overload]{empheq} % left\{ for align % no crash kile ?
\usepackage{cases}
\usepackage[usenames,dvipsnames]{xcolor} % before tikz, options for more colors
\definecolor{light-gray}{gray}{0.95}

\usepackage{booktabs}
\usepackage{caption}
\usepackage{subfig}
% \captionsetup[figure]{width=.85\textwidth}
\captionsetup[subfigure]{margin=0.5cm}
\usepackage{tikz}
\usetikzlibrary{matrix}
\usetikzlibrary{positioning}
% set arrows as stealth fighter jets
\tikzset{>=stealth}
% bezier
\usetikzlibrary{decorations.pathreplacing}
\tikzset{%
  show curve controls/.style={
    postaction={
      decoration={
        show path construction,
        curveto code={
          \draw [blue] 
            (\tikzinputsegmentfirst) -- (\tikzinputsegmentsupporta)
            (\tikzinputsegmentlast) -- (\tikzinputsegmentsupportb);
          \fill [red, opacity=0.5] 
            (\tikzinputsegmentsupporta) circle [radius=.5ex]
            (\tikzinputsegmentsupportb) circle [radius=.5ex];
        }
      },
      decorate
}}}
\tikzstyle{mybox} = [draw=gray, fill=light-gray, very thick,
    rectangle, rounded corners, inner sep=10pt, inner ysep=20pt]
\tikzstyle{mytitle} =[fill=gray, text=white]

% plots
\usepackage{pgfplots}

%\usepackage{color}

\usepackage{xcolor}
\definecolor{bookColor}{cmyk}{1	, 1  , 0   , 0}  % 0.90\% of black
%\color{bookColor}

\usepackage{hyperref}
\hypersetup{pdftex,colorlinks=true,allcolors=blue}
\usepackage{hypcap}
\usepackage{etoolbox}

\usepackage{csquotes}
%\usepackage[autostyle, italian=guillemets]{csquotes}

% \usepackage[chapter]{placeins} % floatbarrier

\usepackage[backend=biber, style=alphabetic]{biblatex}
\addbibresource{sources.bib}
\DeclareFieldFormat[article]{title}{\textit{#1}}

\numberwithin{equation}{section}

\theoremstyle{definition}
\newtheorem{definition}[subsection]{Definition}
\theoremstyle{plain}
\newtheorem{theorem}[subsection]{Theorem}
\theoremstyle{plain}
\newtheorem{corollary}[subsection]{Corollary}
\theoremstyle{plain}
\newtheorem{proposition}[subsection]{Proposition}
\theoremstyle{plain}
\newtheorem{remark}[subsection]{Remark}
\theoremstyle{plain}
\newtheorem{lemma}[subsection]{Lemma}
\theoremstyle{plain}
\newtheorem{example}[subsection]{Example}

\theoremstyle{plain}
\newtheorem{assumption}[subsection]{Assumption}
\theoremstyle{plain}
\newtheorem{problem}[subsection]{Problem}

\DeclareMathOperator{\divergence}{div}
\DeclareMathOperator{\curl}{curl}
\DeclareMathOperator{\real}{Re}
\DeclareMathOperator{\imag}{Im}

\providetoggle{verbose}
\settoggle{verbose}{true}
\providetoggle{old}
\settoggle{old}{false}
\providetoggle{fig}
\settoggle{fig}{true}
% \toggletrue{paper}
% \togglefalse{paper}
\renewcommand{\i}{\textup{i}}
\let\phi\varphi
\let\epsilon\varepsilon

\newcommand{\upin}{\textup{ in }}
\newcommand{\upon}{\textup{ on }}

%%%%%%%%%%%%%%%%%%%%%%%%%%%%%%
\usepackage{amssymb,tikz}

\newcommand{\mysetminusD}{\hbox{\tikz{\draw[line width=0.6pt,line cap=round] (3pt,0) -- (0,6pt);}}}
\newcommand{\mysetminusT}{\mysetminusD}
\newcommand{\mysetminusS}{\hbox{\tikz{\draw[line width=0.45pt,line cap=round] (2pt,0) -- (0,4pt);}}}
\newcommand{\mysetminusSS}{\hbox{\tikz{\draw[line width=0.4pt,line cap=round] (1.5pt,0) -- (0,3pt);}}}

\newcommand{\mysetminus}{\mathbin{\mathchoice{\mysetminusD}{\mysetminusT}{\mysetminusS}{\mysetminusSS}}}
%%%%%%%%%%%%%%%%%%%%%%%%%%%%%%%%%%


\usepackage{environ}
\NewEnviron{mybox}{%
\begin{center}
\colorbox{light-gray}{\color{black}\parbox{\textwidth}{%
% \fcolorbox{gray}{light-gray}{
\BODY
}}
\end{center}
}

\usepackage{fancyhdr}
\newcommand{\fncyblank}{\fancyhf{}}
%%%%%%%%%%%%%%%%%%%%%%%%%%%%%%%%%%%%%%%%%%%% abstract already defined
% \newenvironment{abstract} %
% {\cleardoublepage\fncyblank\null\vfill\begin{center} %
% \bfseries\abstractname\end{center}} %
% {\vfill\null}
%%%%%%%%%%%%%%%%%%%%%%%%%%%%%%%%%%%%%%%%%%%%%%%%%%%%%%%%%%%%%%%%%%%%%%%%%
\title{The Boundary Integral Method}
\author{Giacomo Milan}
% \date{3 Ottobre 2017} % oppure anno accademico
% \usepackage[]{frontespizio}

\begin{document}
% \frontmatter
% \begin{frontespizio}
% \Universita{Politecnico di Milano}
% \Facolta{Scienze Matematiche, Fisiche e
% Naturali}
% \Corso[Laurea]{Matematica}
% \Titoletto{Project of the course \\ Advanced Numerical Partial Differential Equations}
% \Titolo{The Boundary Integral Method}
% \Candidato[145822]{Giacomo Milan}
% % \Relatore{Ch.mo Prof.~Adalberto Orsatti}
% \Annoaccademico{2016-2017}
% \end{frontespizio}
\begin{titlepage}
    \begin{center}
        \Large
        \vspace*{1cm}
        \textsc{Politecnico di Milano}\\
        Scuola di Ingegneria Industriale e dell'Informazione\\
        \vspace{1cm}
        \includegraphics[width=0.5\textwidth]{logo_bw}
        
%       \vfill
        \normalsize
        \vspace{1cm}
        Project of \\
        \large
        \textsc{Numerical Analysis for Partial Differential Equations}\\
        \begin{flushright}
        \normalsize
         Prof.ssa Perotto
        \end{flushright}
%         \vspace{0.5cm}
        \normalsize
        Course of \\
        \large
        \textsc{Mathematical Engineering}
        

        \vspace{0.8cm}
        \huge
        \textsc{The Boundary Element Method}
        
%       \vspace{0.5cm}
%       Thesis Subtitle
        
        \vspace{1.5cm}
        \begin{flushright}
        \normalsize
        {Giacomo Milan, matr. 841530}         
        \end{flushright}
        \normalsize
        \vfill
        Anno accademico 2016-2017
        
    \end{center}
\end{titlepage}
% \clearpage
% \cleardoublepage
\newpage
\thispagestyle{empty}
\mbox{}
\newpage
% \maketitle
\tableofcontents
% \listoffigures
%\listoftables
% \mainmatter
\clearpage

\section{Introduction}
The theoretical origin of the Boundary Element Method can be traced to the nineteenth century,
when many applications given by the electrostatics, governed by the Laplace equation, 
led to the development of the \emph{potential theory} 
and to integral potentials to solve boundary value problems
(for a detailed early history see \cite{cheng-cheng:history}).
All the greatest mathematicians gave their contribution to the integral theory, 
and later to Green's representation formulas, that found many concrete examples in 
elasticity, fluid  dynamics, acoustic problems.
\par
In the pre-electronic computing era, W. Ritz proposed the representation of the solution
$u(z)=\sum\alpha_i\psi_i(z)$ as the sum of trial functions defined in subdomains, a method 
that led to FEM. The same approach yielded to the representation 
$u(z)=\sum\alpha_i\Phi(z,z_i)$ called the \emph{method of fundamental solutions}, with sources
and sinks collocated in the points $z_i$, distributed on the boundary, or outside the domain.
The problem was complete with the imposition of the boundary data in the collocation
nodes.
\par
In the electronic computing era we can remember M. A. Jaswon which implemented first discretizations 
of integral Green's formulas, and V. D. Kupradze, whose approach, called \emph{method of 
functional equations}, was to represent the solution through auxiliary contours, distinct 
from the domain's boundary, with the advantage of smooth integral kernels.
The definitive turning point to \emph{boundary integral equation} BIE and 
\emph{boundary element method} BEM were given by F. J. Rizzo, T. A. Cruse and 
C. A. Brebbia who talked about these ideas in 1977 at the First International
Symposium on Innovative Numerical Analysis in Applied
Engineering Sciences, at Versailles, France, 
and in 1978 published the first textbook on BEM
‘The Boundary Element Method for Engineers' \cite{brebbia:book}.
In section \ref{section:bie} we will present the derivation of Boundary Integral Equations, 
and in section \ref{section:bem} we will show the discretization algorithm called Boundary
Element Method. In \ref{section:bemfem} we compare this method with the Finite Element Method
and then we will show some numerical results in section \ref{section:num}.
\section{Boundary Integral Equations}
\label{section:bie}
In this section we introduce the theory of Boundary Integral Equations BIE,
starting from an example of boundary 
value problem BVP. We consider the Laplace equation in an open set $\Omega$ of class $C^2$
\begin{eqnarray}
 \Delta u = 0, & \textup{ in }\Omega,\label{eq:laplace-in}\\
 u = g, & \textup{ on }\partial\Omega.\label{eq:laplace-on}
\end{eqnarray}
The aim is to represent the solution $u$ through \emph{representation formulas} 
in terms of its boundary data $u|_{\partial \Omega}$, $\partial_\nu u|_{\partial \Omega}$, and 
of $\Delta u$.
\par
We generalize the previous example to the problem $Lu = f$ in $\mathcal{D}'(\Omega)$, in the 
distributional sense, denoting by $L$ a differential operator with smooth coefficients.
Let $\Phi(z,z_0)$ denote the \emph{fundamental solution} which solves
\begin{equation}
\label{eq:fundamental-solution}
 L\Phi=\delta_{z_0},\quad\upin\mathcal{D}'(\mathbb{R}^m).
\end{equation}
For the Laplace equation with $L=-\Delta$, it can be explicitly expressed as
\begin{equation}
\label{eq:definition-Phi-23}
  \Phi(z,z_0)=
  \left\{
  \begin{aligned}
   &\dfrac{1}{2\pi}\log\dfrac{1}{| z - z_0|}, && m=2, \\
   &\dfrac{1}{4\pi}\dfrac{1}{| z  - z_0|}, && m=3.
  \end{aligned}
  \right.
\end{equation}
Let $\Omega$ be a domain of class $C^2$ (or even Lipschitz regular),
and $u \in C^2(\Omega)\cap C^1(\overline{\Omega})$,
then, by classical Green's formula,
there holds the following integral representation formula
\begin{equation}
  \label{eq:representation-formula}
  - u(z_0) = \int_\Omega\Delta u\,\Phi_{z_0}\,dz 
  + \int_{\partial \Omega}\big(u\, \partial_{\nu(z)} \Phi_{z_0}
  - \partial_\nu u\,\Phi_{z_0}\big)\,dz, \quad z_0 \in \Omega.
\end{equation}
% \begin{proposition}[Classical Green's Representation]
%  Let $D$ be a Lipschitz domain and $u,v \in C^2(D)\cap C^1(\overline{D})$ be regular functions, then integration by parts can be expressed as
%   \begin{equation}
%   \int_D (u \, \Delta v - \Delta u \, v)\, dx = \int_{\partial D}(u \, \partial_\nu v - \partial_\nu u \, v)\,d\sigma = \mathcal{R}_{\partial D}(u,v).
%   \end{equation}
% \end{proposition}
% \begin{proposition}[Representation of harmonics]
%  Indeed formally, for distributions in $\mathbb{R}^m$ which take to account the boundary of $D$, $\langle u,\Delta_p\Phi(p,p_0)\rangle = \langle u,-\delta_{p_0}\rangle = - u(p_0)$.
% \end{proposition}
In our case, any solution $u$ of \eqref{eq:laplace-in} can be represented
\begin{equation}
  \label{eq:representation-formula-harmonics}
  - u(z_0) = \int_{\partial \Omega}\big(u\, \partial_{\nu(z)} \Phi_{z_0}
  - \partial_\nu u\,\Phi_{z_0}\big)\,dz, \quad z_0 \in \Omega,
\end{equation}
as the sum of two layer potentials defined on $\partial\Omega$.
Given a density $\psi\in C(\partial \Omega)$, we define
\begin{enumerate}
  \item the \emph{single layer} potential
   \begin{equation}
    \mathcal{S}(\partial \Omega,\psi)(z_0)\coloneqq \int_{\partial \Omega} \Phi(z_0, z)\psi(z)\, dz,\quad z_0\in\mathbb{R}^m \backslash\partial \Omega, \label{eq:definition-single-layer}
   \end{equation}
  \item the \emph{double layer} potential
   \begin{equation}
    \mathcal{D}(\partial \Omega,\psi)(z_0)\coloneqq \int_{\partial \Omega} \partial_{\nu(z)}\Phi(z_0, z)\psi(z)\, dz,\quad z_0\in\mathbb{R}^m \backslash\partial \Omega. \label{eq:definition-double-layer}
   \end{equation}
\end{enumerate}
These potentials are analytic in $\mathbb{R}^m\backslash\partial\Omega$,
while along the boundary $\partial\Omega$ there hold the well known \emph{jump relations}, 
which can be expressed through the integral operators 
$S, K, K': C(\partial \Omega)\to C(\partial \Omega)$.
\begin{definition}
 We denote by $S$, $K$  and $K'$ the following integral operators 
 defined on $C(\partial \Omega)$, where $\partial \Omega$ is of class $C^2$
 \begin{enumerate}
  \item  $ S: C(\partial \Omega) \to C(\partial \Omega)$
  \begin{equation}
  S\psi(z_0)\coloneqq\int_{\partial \Omega}\psi(z) \Phi(z_0,z)\,dz,\label{def:operator-S}
  \end{equation}
  \item  $ K,K':C(\partial \Omega) \to C(\partial \Omega)$
  \begin{align}
  & K\psi(z_0)\coloneqq\int_{\partial \Omega} \psi(z) \partial_{\nu(z)} \Phi(z_0, z) \,dz =
  \int_{\partial \Omega} \psi(z) \nabla_z\Phi(z_0, z)\cdot\nu(z) \,dz,\label{def:operator-K}\\
  & K'\psi(z_0)\coloneqq\int_{\partial \Omega} \psi(z) \partial_{\nu(z_0)} \Phi(z_0, z) \,dz =
  \int_{\partial \Omega} \psi(z) \nabla_{z_0}\Phi(z_0, z)\cdot\nu(z_0) \,dz.\label{def:operator-K'}
  \end{align}
 \end{enumerate}
\end{definition}
\begin{theorem}
\label{theo:jump-relations}
 Let $\psi\in C(\partial \Omega)$, $\partial \Omega$ be of class $C^2$, and let 
 $\mathcal{S}(\psi,\partial \Omega)$, $\mathcal{D}(\psi,\partial \Omega)$ be the potentials 
 defined in \eqref{eq:definition-single-layer} and \eqref{eq:definition-double-layer}, then
 \begin{enumerate}
  \item the single layer is continuous and 
  \begin{subequations}
  \begin{align}
   \mathcal{S}^\pm(z) &\coloneqq\lim_{h\to 0^\pm}\mathcal{S}(z+h\nu(z)) = S\psi(z)=\int_{\partial \Omega}\psi(y)\Phi(z,y)\,dy \quad z\in\partial \Omega, \label{eq:single-pm-0}\\
   \partial_\nu\mathcal{S}^\pm(z) &\coloneqq \lim_{h\to0^\pm} \nabla\mathcal{S}(z+h\nu(z))\cdot\nu(z) =  K'\psi(z) \,\mp\,\dfrac{1}{2}\psi(z) \quad z\in\partial \Omega,\label{eq:single-pm-1}
  \end{align}
 \end{subequations}
 \item the double layer can be continuously extended from $\Omega$ to $\overline{\Omega}$, from $\mathbb{R}^m\backslash \overline{\Omega}$ to $\mathbb{R}^m\backslash \Omega$, and
  \begin{subequations}
  \begin{align}
   \mathcal{D}^\pm(z) &= K\psi(z) \pm\dfrac{1}{2}\psi(z)\quad z\in\partial \Omega, \label{eq:double-pm-0}\\
   \partial_\nu\mathcal{D}^+(z) &= \partial_\nu\mathcal{D}^-(z) \quad z\in\partial \Omega. \label{eq:double-pm-1}
  \end{align}
  \end{subequations}
 \end{enumerate}
\end{theorem}
% For a domain $\Omega$ of class $C^2$, there hold the well known \emph{jump relations}
% which can be written as
% \begin{equation}
% \label{eq:jump-relations}
%  [\partial_\nu\mathcal{S}]_{\partial \Omega} = -\psi,
%  \quad
%  [\mathcal{D}]_{\partial \Omega} = \psi.
% \end{equation}
We remark that they are well defined for a regular 
boundary $\partial\Omega$, thanks to the non trivial convergence of the integral of 
the singularity contained in the fundamental solution $\Phi$.
The derivation of representation formulas can be done more rigorously in the 
distributional framework, introducing the notion of pseudo-differential operators.
\par
The solution of problem \eqref{eq:laplace-in}--\eqref{eq:laplace-on}, with Dirichlet data
$g$, through representation \eqref{eq:representation-formula-harmonics} containing 
the unknown $\partial_\nu u|_{\partial\Omega}$, can proceed in two ways
\begin{enumerate}
 \item eliminating the dependence of $\partial_\nu u$, constructing a representation
 formula with the \emph{Green function} for the homogeneous Neumann problem in $\Omega$, 
 that is
  \begin{alignat}{2}
  -\Delta G &= \delta_{z_0}, && \textup{ in }\Omega,\label{eq:green-neum-in}\\
  \partial_\nu G &= 0, && \textup{ on }\partial\Omega;\label{eq:green-neum-on}
  \end{alignat}
  \item finding an equation for $\partial_\nu u$. 
\end{enumerate}
We will follow the second approach, taking advantage of jump relations previously stated,
to derive a \emph{boundary integral equation}.
If we consider the Laplace problem \eqref{eq:laplace-in}, it's convenient to represent the 
solution in different ways, according with the nature of boundary conditions
\begin{align}
  u = g, & \textup{ on }\partial\Omega \quad \Rightarrow \quad u = \mathcal{D}(\partial\Omega;\, \psi), \label{eq:bc-lap-dir}\\
  \partial_\nu u = g, & \textup{ on }\partial\Omega \quad \Rightarrow \quad u = \mathcal{S}(\partial\Omega;\, \psi), \label{eq:bc-lap-neum}
\end{align}
which respectively yield to equations
\begin{align}
 (K - 0.5 I)\psi &= g,\label{eq:bie-dir}\\
 (K' + 0.5 I)\psi &= g.\label{eq:bie-neum}
\end{align}

Both the equations \eqref{eq:bie-dir} and \eqref{eq:bie-neum} are in the form of an \emph{integral equation 
of the second kind} $I - \tilde{K}:X\to Y$, where $\tilde{K}$ is a linear compact operator, between Banach spaces.
This entails well position of the linear equation and the existence of a bounded inverse operator.
According with the Boundary Element Method, we consider the discretization
of $A\psi=f$ in $Y$, and we solve it for $u$ belonging to the finite dimensional 
subspace $X_n\subset X$.
\par
We denote by $\{\psi_j\}_{i=0}^m$ the finite basis of $X_n$.
Therefore the integral equations \eqref{eq:bie-dir} and \eqref{eq:bie-neum} are reduced to
\begin{equation}
 \sum_{j=0}^m(A\psi_j) \alpha_j = f \upon \partial\Omega,\label{eq:directlinear}
\end{equation}
where, in the considered interior problem with Dirichlet data, the full expression of $A\psi_j$ is
\begin{equation}
 A\psi_j(z_0) = \int_{\partial\Omega}\partial_{\nu(z)}\Phi(z_0, z)\psi_j(z)\,dz  - \frac{1}{2}\psi_j(z_0).
\end{equation}
There exist several methods to solve the equation \eqref{eq:directlinear}, with $\psi\in X_n$, 
for a linear operator $A:X\to Y$. They 
can be classified under the name of \emph{projection methods} (see \cite{kirsch:book}), since
they project the equation on a finite dimensional subspace $Y_n\subset Y$.
The most common methods are the \emph{Galerkin method}, which considers the orthogonal projection,
and the \emph{least squares method}. For equations which derive from boundary integrals, 
the most popular choice is the \emph{collocation method}, which has the advantage of 
performing only one integration, in contrast with the other ones.
\par
We denote by $z_i = c(\tau_i)$ the collocation nodes, which we will take corresponding to 
mesh vertices. Then we impose the boundary data in correspondence of these nodes, that is
\begin{equation}
 \sum_{j=0}^m(A\psi_j)(z_i) \alpha_j = f(z_i), \quad \textup{ or }\quad \sum_{j=0}^mA_{ij}\alpha_j = \beta_i, 
 \quad i=0,\dots,m.\label{eq:bie-collocated}
\end{equation}
We are left to choose a quadrature rule which doesn't affect the order of convergence, 
in the computation of the matrix's element
\begin{equation}
\label{eq:Aij}
 A_{ij} = \int_{\partial\Omega}\partial_{\nu(z)}\Phi_{z}(z_i) \psi_j(z)\,dz - \frac{1}{2} \psi_j(z_i).
\end{equation}
For a boundary $\partial\Omega$ of class $C^2$, the integral kernel $\partial_{\nu(z)}\Phi_{z}(z_i)$ is 
continuous (there exists a finite limit for $z\to z_i$), but in general the integral 
kernel $K(z_i,z)$ has a singularity.
\par 
In the sequel, we will consider the fundamental solutions
of the Laplace equation and the Helmholtz equation. Since both of them have a logarithmic 
singularity, the most efficient method to approximate the improper integral 
is to not change the quadrature nodes (we usually consider the equidistant nodes, 
without recomputing them, for all the collocation points)
and to rewrite the kernel as the sum of two parts. 
We fix the notation denoting by $M(z_0,z)$ the integral kernel of the single 
layer operator $S$, and by $L(z_0, z)$ the kernel of the double layer operator $K'$, 
with $z=c(\tau)$, $z_0=c(t)$, both belonging to $\partial\Omega$
\begin{align}
 L(t,\tau)
 \coloneqq
%  &\frac{1}{2\pi}\frac{1}{|c(t)-c(\tau)|}\frac{c(t)- c(\tau)}{|c(t)-c(\tau)|}
%  \cdot \nu(c(\tau))|c'(\tau)|\\
%  =
 &L_1(t,\tau)\ln\Big(4\sin^2\frac{t-\tau}{2}\Big)+L_2(t,\tau),
 \label{eq:def-kernel-L}\\
 M(t,\tau)
 \coloneqq
 &M_1(t,\tau)\ln\Big(4\sin^2\frac{t-\tau}{2}\Big)+M_2(t,\tau),
 \label{eq:def-kernel-M}
\end{align}
% \begin{equation}
% \label{eq:def-kernel-M}
%  M(t,\tau)\coloneqq-\frac{1}{2\pi}\ln|c(t)-c(\tau)|=M_1(t,\tau)\ln\Big(4\sin^2\frac{t-\tau}{2}\Big)+M_2(t,\tau),
% \end{equation}
where all $L_1(t,\tau)$, $L_2(t,\tau)$, $M_1(t,\tau)$, $M_2(t,\tau)$ turn out
to be continuous and bounded (actually they are all analytic).
It's possible to compute the integral of the logarithmic singularity
through the trigonometric interpolation of the integrand function.
Indeed there hold the following orthogonality relations
(see Lemma 8.21 contained in \cite{kress:book}),
\begin{equation}
 \frac{1}{2\pi}\int_0^{2\pi}\ln\Bigl(4\sin^2\frac{t}{2}\Bigr)e^{imt}\,dt=
 \begin{cases}
  0, & m=0, \\
  -1/|m|, & m=\pm1,\pm2,\dots
 \end{cases}
\end{equation}
This result let us to compute the exact weights 
\begin{equation}
 R_j^{(n)}(t)=\frac{1}{2\pi}\int_0^{2\pi}\ln\Bigl(4\sin^2\frac{t-\tau}{2}\Bigr)L_j(\tau)d\tau, \quad j=0,\dots,2n-1,
\end{equation}
from the integral of $L_j(t)$, denoting the function of the trigonometric lagrangian basis.

% \begin{equation}
%  R_j^{(n)}(t)=-\frac{1}{n}\Bigl\{\sum_{m=1}^{n-1}\frac{1}{m}\cos m(t-\tau_j) + \frac{1}{2n}\cos n(t-\tau_j)\Bigr\}, \quad j=0,\dots,2n-1.
% \end{equation}

\section{The Boundary Element Method}
\label{section:bem}
In this section we will present the discretization method called Boundary
Element Method, whose details are for example contained in 
\cite{brebbia:book}, \cite{brebbia:book-progress}, applied
in particular to elasticity problems and to mechanics.
\par
Let $\partial\Omega$ be a curve 
in $\mathbb{R}^2$ of class $C^2$ parametrized by
\begin{equation}
c(t):[0,2\pi]\to\mathbb{R}^2,
\end{equation}
and let $\mathcal{P}_n = \{[\tau_k^{(n)},\tau_{k+1}^{(n)}]\}_{k=0}^{2n-1}$
be a partition of 
the interval $[0, 2\pi]$, with equidistant nodes $\tau_k^{(n)}=k\pi/n$, 
which defines a partition $\mathcal{E}_n = \{E_k\}_{k=0}^{2n-1}$ of the boundary 
$\partial \Omega=\bigcup E_k$, with $E_k \coloneqq c([\tau_k, \tau_{k+1}])$.
\begin{center}
\begin{tikzpicture}
\draw [thick] (-2, -1) 
  .. controls ++(80:1) and ++(20:-0.8) .. (0, 1.5)
  .. controls ++(20:0.8) and ++(120:1) .. (3, 0);
\node at (-2,-1.2){$c(\tau_{k+2})$};
\node at (-0.6,1.7){$c(\tau_{k+1})$};
\node at (3.45,0){$c(\tau_{k})$};
\node at (1.5,0.5){$e_k$};
\node at (1.5,1.6){$E_k$};
\node at (-0.6,-0.1){$e_{k+1}$};
\node at (-1.9,0.5){$E_{k+1}$};
\node at (3.0,1.5)(boundary){$\partial \Omega$};

\draw [fill] (-2,-1) circle [radius=1pt];
\draw [fill] (0,1.5) circle [radius=1pt];
\draw [fill] (3,0) circle [radius=1pt];
\draw [] (-2,-1)--(0, 1.5);
\draw [] (3,0)--(0, 1.5);
\end{tikzpicture}
\hspace*{-1cm}
\begin{tikzpicture}
\draw [] (-1,0)--(1, 0);
\draw [fill] (-1,0) circle [radius=1pt];
\draw [fill] (0,0) circle [radius=1pt];
\draw [fill] (1,0) circle [radius=1pt];
\node at (-1,-0.2){$\tau_{k}$};
\node at (0,-0.2){$\tau_{k+1}$};
\node at (1,-0.2){$\tau_{k+2}$};

\node (x2) at (-4,2.5){$ $};
\node (x1) at (0,0.5){$ $};

\draw[->] (x1) to [out=90,in=0] node[above,midway]{$c$}(x2);
\end{tikzpicture}
\end{center}
Now we introduce the finite element spaces, at the same way they 
are defined for a two or three dimensional mesh $\mathcal{T}_h$,
in other methods, like FEM
(see \cite{Quarteroni:book}).
Let $\partial\Omega_h \coloneqq \bigcup e_k\approx\partial\Omega$, 
with $e_k = [c(\tau_k), c(\tau_{k+1})]$, be the approximate boundary,
then
we consider the finite element space of continuous, piecewise polynomial functions of degree $r$,
constructed on the discretized boundary $\partial\Omega_h$, that is
\begin{equation}
 X^r_h \coloneqq \{v_h\in C^0(\partial\Omega_h):v_h|_{e_k} \in \mathbb{P}^r,\, \forall e_k\in \partial\Omega_h\}.
\end{equation}
We can associate to $X^r_h$ a collection of nodes, denoted in general by $z_i$ and 
coincident with collocation nodes used for the boundary integral equation \eqref{eq:bie-collocated}, 
such that we can define a \emph{lagrangian basis} for $X^r_h$,
denoted by $\{\psi_j\}_{i=0}^m$.
\par
An alternative finite dimensional space for regular domains, 
when it's given the explicit parametrization of $\partial \Omega$,
is the space of finite element functions $v(c(t))$ on $[0,2\pi]$.
We denote by $X_n$ the space constructed as before
for a given partition $\mathcal{P}_n$ of $[0, 2\pi]$, that is
\begin{equation}
 X^r_n \coloneqq \{v_n\in C^0(\partial\Omega):v_n(c(t))|_{I_K} \in \mathbb{P}^r,\, \forall I_K\in \mathcal{P}_n\}.
\end{equation}
\par
In some applications it's admissible the choice of piecewise constant boundary
element functions, denoted by $X^0_h$. Consequently the domain of the integral $A_{ij}$,
with kernel $a$, coincides with the segment $e_j$
\begin{equation}
\label{eq:Aij-X0h}
 A_{ij} = \int_{e_j}a(z_i,z)\,dz.
\end{equation}
For BIE of the second kind, like \eqref{eq:bie-dir}, \eqref{eq:bie-neum},
we can not implement the $X^0_h$ discretization, otherwise the discrete function $\psi(z)$
would be discontinuous in the collocation nodes.
\par
% According with the space of element functions, we choose the best 
% quadrature rule to integrate $A_{ij}$ \eqref{eq:Aij}.
% In many cases, to obtain an high order for the convergence, 
% from the knowledge of fundamental solution, \eqref{eq:Aij-X0h} is
% integrated exactly.
A good compromise is to choose a quadrature rule with one or two points per
segment. Indeed a further refinement of the quadrature mesh would make the method less
competitive with respect to the same method with more nodes.

We report the quadrature points and quadrature weights for the integration 
along a one dimensional discrete curve, which are
described in \cite{FreeFem-doc}, documentation of the software
FreeFem++ \cite{FreeFem}, and are implemented in the finite element spaces
denoted by P0, P1, P2.
% 9.6884482205476277804
% wh
% 9.682030647015111
% \begin{table}
% \caption{.}
% \label{tab:ellipse-parameters}
\begin{center}
% \vspace*{-0.5cm}
\begin{tabular}{lccc}
\toprule
 qfe & point in $[z_i, z_{i+1}](=t)$ & 
 $\omega$ & $r$(exact on $\mathbb{P}^r$)\\
\midrule
qf1pE & $1/2$ &  $|z_iz_{i+1}|$ & 1\\
qf2pE & $(1\pm\sqrt{1/3})/2$ &  $|z_iz_{i+1}|/2$ & 3\\
qf1pElump & $\{0, 1\}$ & $|z_iz_{i+1}|/2$ & 1\\
\bottomrule
\end{tabular}
\end{center}
% \end{table}
Therefore the choice of ``qf1pElump'' for $X^1_h$ yields to (by $\psi_j(z_k) = \delta_{jk}$)
% ``‘qf1pElump'''
\begin{equation}
\label{eq:Aij-X1h}
 A_{ij} = \frac{1}{2}\big(|z_jz_{j-1}| + |z_jz_{j+1}|\big)\partial_{z}\Phi(z_i,z_j)\cdot \nu(z_j) - \frac{1}{2} \delta_{ij}.
\end{equation}
The quadrature nodes can be better observed in Figure \ref{fig:qf2}.
We remark that for ``qf1pE'', ``qf2pE'', the singularity it's never evaluated. Instead for ``qf1pElump''
the collocation nodes and the quadrature nodes coincide, and the singularity can appear.
In the above equation \eqref{eq:Aij-X1h} the kernel of $K$ has a finite limit for smooth boundaries (see next section),
but we cannot expect the same for the fundamental function of the Helmholtz equation.
\par
If we choose to integrate on the exact boundary $\partial\Omega$,
we can consider the space $X^1_n$, equivalent to $X^1_h$, with the
the weights $\omega$ substituted by the speed $|c'(\tau)|$.
Furthermore, decompositions \eqref{eq:def-kernel-L} and \eqref{eq:def-kernel-M} let us to integrate even singular kernels.
% \begin{equation}
% \label{eq:Aij-X1n}
%  A_{ij} = \int_{\partial\Omega}\partial_{\nu(z)}\Phi_{z}(z_i) \psi_j(z)\,dz
% \end{equation}
The choice of $X^1_n$ is particular competitive for smooth domains, 
that can be easily parametrized.
% Therefore separating the singularity, we can approximate \eqref{eq:Aij-X1n} 
% without refining the quadrature nodes

\section{BEM vs FEM}
\label{section:bemfem}
% We are advantaged by the reduced amount of computation, since Nystr\"{o}m method requires only 
% one integration, while two integrations are needed for the Galerkin method, and three 
% integrations for the Least Squares method.
We list main advantages of the formulation of the problem through boundary integrals, 
instead of using for instance the discretization of the variational formulation with the FEM.
\begin{enumerate}
 \item The expression of the solution reduces to a boundary integral,
 and allows not to consider the full domain. The discretization of a curve, instead of an area 
 in two dimension (or a surface instead of a volume in three dimension), is less expensive.
 \item We can deal with unbounded regions in exterior problems, as easily as we do with bounded regions.
 \item Most of the times, for smooth boundaries and smooth data, the rate of convergence is very high, even exponential.
\end{enumerate}
On the other end, we present the main disadvantages.
\begin{enumerate}
 \item The boundary integrals formulation requires the knowledge of the fundamental solution. 
 This fact depends on the nature of the differential equation 
 and prevent us from solving 
 differential problem with non constant parameters.
 \item Main difficulties comes from singularities of the integral kernels, 
 which requires a convergent quadrature method.
 \item Some attention must be payed to boundaries with corners, 
 where the unknown densities are singular. The quadrature rule requires a more 
 careful collocation of the nodes.
\end{enumerate}

\section{Numerical examples}
\label{section:num}
In this section we consider mainly the numerical discretization of interior and 
exterior BVPs for the Laplace equation and the Helmholtz equation.
We can implement the same algorithm for both of them, since the jump relations 
in Theorem \ref{theo:jump-relations}
remain unchanged for both equations. Furthermore, their fundamental solutions have
the same kind of singularity, and the only modification is the explicit computation of
the limit values $M_2(t,t)$ and $L_2(t,t)$.
Most of the exact expressions that we will use in the sequel, can be found in 
\cite{colton-kress:book}, and we imported their implementation in the Python package
from the MATLAB toolbox MPSpack \cite{mpspack}. We will present all the examples in two dimension, 
since we can implement them much more easier, even if the theory of boundary integrals 
is true for problems in a three dimensional framework.
\par
We consider the Laplace equation $-\Delta u = 0$ in a open domain $\Omega$ of class $C^2$. 
We start from the knowledge of its fundamental solution \eqref{eq:definition-Phi-23} and
the fact that the integral kernel $L(t,\tau)$ of the double layer operator $K'$ is continuous
for regular domains (the same holds for $K$), with limit value for $\tau \to t$ equal to
\begin{align}
 &\lim_{\tau \to t}L(\tau,t)=\lim_{\tau \to t}\frac{1}{2\pi}\frac{c(t)-c(\tau)}{|c(t) - c(\tau)|^2}\cdot\nu(c(\tau))|c'(\tau)|
%  \\
%  =&\lim_{\tau \to t}\frac{1}{2\pi}\frac{-\frac{1}{2}c''(\xi)|t-\tau|^2\cdot\nu(c(\tau))|c'(\tau)|}{|c'(\zeta)|^2|t-\tau|^2}
%  = -\frac{1}{4\pi}\frac{c''(t)\cdot\nu(t)}{|c'(t)|} 
 = -\frac{1}{4\pi}\kappa(t)|c'(t)|,
\end{align}
On the other end, the integral kernel $M(t,\tau)$ of the single layer operator 
is computed as the sum of two terms
\begin{equation}
\begin{cases}
 M_1(t,\tau)= -1/(4\pi), \\
 M_2(t,\tau)=M(t,\tau) - M_1(t,\tau)\ln\Big(4\sin^2\dfrac{t-\tau}{2}\Big), & t\neq \tau,
\end{cases}
\end{equation}
with a diagonal term computed from the asymptotic expansion
\begin{align}
 M_2(\tau,\tau)&=\lim_{t\to\tau} \Big(M(t,\tau) - M_1(t,\tau)\ln\Big(4\sin^2\frac{t-\tau}{2}\Big)\Big)
%  \\
%  &=2M_1(\tau,\tau)\ln\big(|c'(\tau)|\big)
 =-\frac{1}{2\pi}\ln|c'(\tau)|.
\end{align}
Therefore, in our first implementation, we will use the computation of the
speed $|c'(\tau)|$ and the curvature $\kappa(\tau)$ from the explicit expression
of $c(t)$, $c'(t)$ and $c''(t)$, which will be the parametrization of a circle, or an ellipse.
As already observed in section \ref{section:bie}, we will adopt the most convenient representation according 
with boundary conditions, that will be \eqref{eq:bc-lap-dir} and \eqref{eq:bc-lap-neum} for interior problems. 
\par
We start from the implementation of the space $X^1_n$, and
we choose to approximate the integral in \eqref{eq:Aij} by the trapezoidal quadrature rule, 
taking quadrature nodes coincident with collocation nodes.
We obtained the numerical approximations in Figure \ref{fig:cf_lap_neum_circle}, with the
imposed boundary data $g$ for the Neumann interior problem 
computed from the harmonic function $\tfrac{1}{6}(x^3 - 3xy^2)$.
\par

\begin{center}
\begin{figure}
% \subfloat[][\emph{Linear sampling method for $\alpha=\mathrm{1e}{-10}$}.]
{
\includegraphics[width=.42\textwidth]{fig/cf_lap_neum_int_circle_exact}
}
{
\includegraphics[width=.42\textwidth]{fig/cf_lap_neum_int_circle}
}
\\
{
\includegraphics[width=.42\textwidth]{fig/cf_lap_neum_ext_circle_exact}
}
{
\includegraphics[width=.42\textwidth]{fig/cf_lap_neum_ext_circle}
}
% \subfloat[][\emph{Factorization method}.]
% {
% \includegraphics[width=.48\textwidth]{fig/one_ellipse_fm_ellipse0}
% }
\caption{\emph{Images of the exact solution $g$ on the left, and the approximation $u$ computed on the right,
for the interior and the exterior Laplace problem with Neumann conditions}.}
\label{fig:cf_lap_neum_circle}
\end{figure}
\end{center}

An analysis of the convergence of the collocation method can be found in 
\cite{chen-zhou:book} or in \cite{kress:book}.
In both cases a fundamental role is played by the convergence results of quadrature rules employed.
In Figures \ref{fig:convergence_lap_neum_circle}, \ref{fig:convergence_lap_neum_ellipse} and \ref{fig:convergence_lap_dir_ellipse} 
we reported the estimated order of convergence of the $L^2(\partial\Omega)$ norm of the
error of the trace of the solution. As can be observed, the numerical approximation 
converges with $n$ to the second power to the solution, according with the trapezoidal rule
\begin{equation}
 \|u_n - u\|_{L^2(\partial\Omega)}\leq Cn^{-2}.
\end{equation}

\par
The implementation of the boundary element method for the exterior BVPs
requires only 
a small change in the sign of the jumps in the integrals equations 
\eqref{eq:bie-dir} and \eqref{eq:bie-neum}, 
that is
\begin{align}
 (K + 0.5 I)\psi &= g,\label{eq:bie-dir-ext}\\
 (K' - 0.5 I)\psi &= g.\label{eq:bie-neum-ext}
\end{align}
In Figures \ref{fig:convergence_lap_neum_circle}, \ref{fig:convergence_lap_neum_ellipse} and \ref{fig:convergence_lap_dir_ellipse} we reported the same analysis of convergence which we have done 
previously, with the
imposed boundary data $g$ computed from $  g= - 1/6(x^3 - 3xy^2)/(
(x^3 - 3xy^2)^2 + (3x^2y - y^3)^2)$.

\begin{center}
\begin{figure}
% \subfloat[][\emph{Linear sampling method for $\alpha=\mathrm{1e}{-10}$}.]
{
\includegraphics[width=.40\textwidth]{fig/convergence_lap_neum_int_circle}
}
{
\includegraphics[width=.40\textwidth]{fig/convergence_lap_neum_ext_circle}
}% \subfloat[][\emph{Factorization method}.]
\caption{\emph{Loglog plot of the $L^2(\partial\Omega)$ error for the interior and the exterior 
Laplace problem in a circle with Neumann conditions, respectively. The estimated exponents in the power of $n$ of the error 
are both equal to $-1.9995$.}}
\label{fig:convergence_lap_neum_circle}
\end{figure}
\end{center}

\begin{center}
\begin{figure}
% \subfloat[][\emph{Linear sampling method for $\alpha=\mathrm{1e}{-10}$}.]
{
\includegraphics[width=.40\textwidth]{fig/convergence_lap_neum_int_ellipse}
}
{
\includegraphics[width=.40\textwidth]{fig/convergence_lap_neum_ext_ellipse}
}% \subfloat[][\emph{Factorization method}.]
\caption{\emph{Loglog plot of the $L^2(\partial\Omega)$ error for the interior and the exterior 
Laplace problem in a ellipse with Neumann conditions, respectively. The estimated exponents in the power of $n$ of the error 
are equal to $-1.9996$ and $-1.9933$.}}
\label{fig:convergence_lap_neum_ellipse}
\end{figure}
\end{center}


\begin{center}
\begin{figure}
% \subfloat[][\emph{Linear sampling method for $\alpha=\mathrm{1e}{-10}$}.]
{
\includegraphics[width=.40\textwidth]{fig/convergence_lap_dir_int_ellipse}
}
{
\includegraphics[width=.40\textwidth]{fig/convergence_lap_dir_ext_ellipse}
}% \subfloat[][\emph{Factorization method}.]
\caption{\emph{Loglog plot of the $L^2(\partial\Omega)$ error for the interior and the exterior 
Laplace problem in an ellipse with Dirichlet conditions, respectively. The estimated exponents in the power of $n$ of the error 
are equal to $-1.9996$ and $-1.9933$.}}
\label{fig:convergence_lap_dir_ellipse}
\end{figure}
\end{center}

\par
The other approach is to implement a finite element basis defined on $\partial\Omega_h$ like $X^2_h$,
whose nodes are well described in the explicative Figure \ref{fig:qf2}.
In some aspects the implementation is more difficult, because we have to approximate
some exact quantities, like the curvature, for the collocation midpoints, defined on $\partial\Omega_h$, 
but not on $\partial\Omega$.
In Figure \ref{fig:ploterrorqf2} we can observe the plot of the error computed
with the $L^\infty(\Omega)$ norm, which decrease according with the power law.
\begin{center}
\begin{figure}
% \subfloat[][\emph{Linear sampling method for $\alpha=\mathrm{1e}{-10}$}.]
\centering
{
\includegraphics[width=.40\textwidth]{fig/geometry_qf2}
}
\caption{\emph{Discretized boundary $\partial\Omega_h$, with red '*' as collocation nodes of $X^2_h$, 
corresponding to vertices and midpoints of the segments, and with blue 'o' as quadrature nodes, as in the 
previous table}.}
\label{fig:qf2}
\end{figure}
\end{center}


\begin{center}
\begin{figure}
% \subfloat[][\emph{Linear sampling method for $\alpha=\mathrm{1e}{-10}$}.]
{
\includegraphics[width=.34\textwidth]{fig/ploterrorqf210}
}
{
\includegraphics[width=.30\textwidth]{fig/ploterrorqf250}
}
{
\includegraphics[width=.30\textwidth]{fig/ploterrorqf250_zoom}
}
% \subfloat[][\emph{Factorization method}.]
% {
% \includegraphics[width=.48\textwidth]{fig/one_ellipse_fm_ellipse0}
% }
\caption{\emph{Error difference $u_h - u$ computed with the space $X^2_h$, 
using $n=10$ in the first image, and $n=50$ in the second, as can be observed with by the colorbars, 
for an elliptic domain $\Omega$. The third image is the zoomed picture of the second to 
evidence that in BEM the error is greater on the integration boundary.}}
\label{fig:ploterrorqf2}
\end{figure}
\end{center}
In \cite{kress:book} are presented some estimates for the trapezoidal rule
applied to an analytic integrand $\psi(z)$, which yield to an exponential convergence of the error
\begin{equation}
 \|Q_n\psi - \psi\|_{L^\infty(\partial\Omega)} \leq Ce^{-\sigma n},
\end{equation}
with respect to the number $n$ of quadrature nodes, strictly connected to the size of the discrete
functional space $X_n$. This behavior is effectively observed in the numerical results.
This behavior can be observed in Figure \ref{fig:errorinfinty}. This method results much more
efficient with respect to $X^2_h$, because of the use of much more exact expressions. Furthermore
it's a combination of collocation with quadrature as proved in \cite{kress:book}.
\begin{center}
\begin{figure}
% \subfloat[][\emph{Linear sampling method for $\alpha=\mathrm{1e}{-10}$}.]
% {
% \includegraphics[width=.25\textwidth]{fig/geometry_qf2}
% }
 \subfloat[][ ]
{
\includegraphics[width=.45\textwidth]{fig/convergence_dir_int_ellipse_qf2_infinity_power}
}
 \subfloat[][ ]
{
\includegraphics[width=.45\textwidth]{fig/convergence_dir_int_ellipse_exponential}
}
% \subfloat[][\emph{Factorization method}.]
% {
% \includegraphics[width=.48\textwidth]{fig/one_ellipse_fm_ellipse0}
% }
\caption{\emph{Logarithmic plots of the $L^\infty(\Omega)$ error. In Figure (a) the error 
computed for $X^2_h$ is logarithmic in both the axes, according with a power law;
in Figure (b) the error 
computed for $X^1_n$ is logarithmic in the $y$ axis, according with an exponential law.}}
\label{fig:errorinfinty}
\end{figure}
\end{center}

\par
The second problem that we consider it's the Helmholtz equation 
in an open set $\Omega$ of class $C^2$
\begin{eqnarray}
 -\Delta u - k^2 u= 0, & \textup{ in }\Omega.\label{eq:helmholtz-in}
%  u = g, & \textup{ on }\partial\Omega.\label{eq:helmholtz-on}
\end{eqnarray}
We can end up with it every time we want to compute the time harmonic eigenmodes
$w(x,t)=e^{-i\omega t}u(x)$, with amplitude $u$, of an elastic membrane
which satisfies the wave equation $w_{tt} - c^2 \Delta w=0$. In an homogeneous medium
we can substitute the square of wave number $k^2$ in equation \eqref{eq:helmholtz-in}
with $n k^2$, where $n=c_0^2/c^2$ is the refraction index and $c_0$ is the speed of sound.
\par
The fundamental solution of the Helmholtz equation in $\mathcal{D}'(\mathbb{R}^m)$ is
\begin{equation}
\label{eq:definition-Phi-23-helm}
  \Phi(z,z_0;k)=
  \left\{
  \begin{aligned}
   &\dfrac{i}{4}H_0^{(1)}(k| z - z_0|), && m=2, \\
   &\dfrac{e^{ik| z - z_0|}}{4\pi| z  - z_0|}, && m=3.
  \end{aligned}
  \right.
\end{equation}
where $H_\alpha^{(1)}(z)=J_\alpha(z)+iI_\alpha(z)$ is the Hankel function of the first kind,
and $J_\alpha(z)$, $I_\alpha(z)$, are respectively the Bessel and the Neumann function
of order $\alpha$.
The algorithm is the same of before, with the difference that both integral kernels 
$L(t,\tau)$ and $M(t,\tau)$ are singular, and 
analytic computations are much more difficult. These can be found in \cite{colton-kress:book}
and have been used in the numerical implementation, that are
\begin{align}
 L_1(t,\tau)\coloneqq&\frac{k}{2\pi}(c_2'(\tau), -c_1'(\tau))\cdot (c(t) - c(\tau)) \frac{J_1(kr(t,\tau))}{r(t,\tau)}, \\
 M_1(t,\tau)\coloneqq&-\frac{1}{2\pi}J_0(kr(t,\tau))|c'(\tau)|,
\end{align}
with diagonal terms
\begin{align}
 L_2(t,t)=&-\frac{1}{4\pi}\kappa(t)|c'(t)|, \\
 M_2(t,t)=&\Big\{\frac{i}{2}-\frac{C}{\pi}
 -\frac{1}{2\pi}\ln\Big(\frac{k^2}{4}|c'(t)|^2\Big)\Big\}
 |c'(t)|,
\end{align}
with $C$ denoting Euler's constant.

\par
An other class of problems which we can consider is the scattering problem in $\mathbb{R}^2$.
Let $\Omega$ be an open domain of class $C^2$, we denote by $u_I$ the incident field, 
for example a plane wave $u_I = e^{-ik(\eta_1 x + \eta_2 y)}$ and by $u_S$ the scattered field, such that the
total field
\begin{equation}
 u=u_I + u_S\label{eq:def-total-fiels} 
\end{equation}
satisfies the Helmholtz equation in 
$\mathbb{R}^2\backslash\overline{\Omega}$, with Sommerfeld radiation 
conditions in $\mathbb{R}^2$ at infinity
\begin{align}
 \Delta u + k^2 u &= 0 \quad \upin \mathbb{R}^2\backslash\overline{\Omega},\\
 u(x)&=O(|x|^{-1}) \quad |x|\to \infty,\\
 \partial_r u - ik u &=o(|x|^{-1})\quad |x|\to \infty,   
\end{align}
and boundary conditions
\begin{enumerate}
 \item homogeneous Dirichlet boundary conditions for a \emph{sound-soft obstacle}
 \begin{equation}
  u = u_I + u_S= 0\quad\upon \partial \Omega,
 \end{equation}
 \item homogeneous Neumann boundary conditions for a \emph{sound-hard obstacle}
 \begin{equation}
  \partial_\nu u = \partial_\nu u_I + \partial_\nu u_S= 0\quad\upon \partial \Omega.
 \end{equation}
\end{enumerate}
The most convenient representation for the scattered field suggested in \cite{chen-zhou:book} is
\begin{equation}
 u_S(z_0)=\int_{\partial \Omega}\Big\{\partial_{\nu(z)}\Phi_z(z_0;k) -i\Phi_z(z_0;k)\Big\}\psi(z)\,dz.
\end{equation}
In Figures \ref{fig:scatt_soft_ellipse_10}, \ref{fig:scatt_soft_ellipse_10_imag}, \ref{fig:scatt_soft_ellipse_3}, 
we can see the approximation computed for the scattering problem with sound-soft boundary conditions 
on $\partial\Omega$.

\begin{center}
\begin{figure}
% \subfloat[][\emph{Linear sampling method for $\alpha=\mathrm{1e}{-10}$}.]
{
\includegraphics[width=.30\textwidth]{fig/scatt_soft_inc_ellipse_10}
}
{
\includegraphics[width=.30\textwidth]{fig/scatt_soft_scatt_ellipse_10}
}
{
\includegraphics[width=.30\textwidth]{fig/scatt_soft_tot_ellipse_10}
}
% \subfloat[][\emph{Factorization method}.]
% {
% \includegraphics[width=.48\textwidth]{fig/one_ellipse_fm_ellipse0}
% }
\caption{\emph{Scattering problem, with homogeneous sound-soft conditions $u=0$. Images of the 
real part
of the incident field $u_I$ (a plane wave with $\eta=(\cos\alpha,\, \sin\alpha)$ for $\alpha=\pi/4$), the scattered field 
$u_S$, and the total field $u=u_I + u_S$, respectively, for $k=10$}.}
\label{fig:scatt_soft_ellipse_10}
\end{figure}
\end{center}
\begin{center}
\begin{figure}
% \subfloat[][\emph{Linear sampling method for $\alpha=\mathrm{1e}{-10}$}.]
{
\includegraphics[width=.30\textwidth]{fig/scatt_soft_inc_ellipse_10_imag}
}
{
\includegraphics[width=.30\textwidth]{fig/scatt_soft_scatt_ellipse_10_imag}
}
{
\includegraphics[width=.30\textwidth]{fig/scatt_soft_tot_ellipse_10_imag}
}
% \subfloat[][\emph{Factorization method}.]
% {
% \includegraphics[width=.48\textwidth]{fig/one_ellipse_fm_ellipse0}
% }
\caption{\emph{Images of the imaginary part of the fields in 
Figure \ref{fig:scatt_soft_ellipse_10}}.}
\label{fig:scatt_soft_ellipse_10_imag}
\end{figure}
\end{center}

\begin{center}
\begin{figure}
% \subfloat[][\emph{Linear sampling method for $\alpha=\mathrm{1e}{-10}$}.]
{
\includegraphics[width=.30\textwidth]{fig/scatt_soft_inc_ellipse_3}
}
{
\includegraphics[width=.30\textwidth]{fig/scatt_soft_scatt_ellipse_3}
}
{
\includegraphics[width=.30\textwidth]{fig/scatt_soft_tot_ellipse_3}
}
% \subfloat[][\emph{Factorization method}.]
% {
% \includegraphics[width=.48\textwidth]{fig/one_ellipse_fm_ellipse0}
% }
\caption{\emph{Images of the real part of the fields $u_I$, $u_S$, $u$, of the same scattering 
problem of Figure \ref{fig:scatt_soft_ellipse_10}, for $k=3$}.}
\label{fig:scatt_soft_ellipse_3}
\end{figure}
\end{center}
\clearpage

\section{Conclusion}
As the final section of this work, we can say that the BEM is very competitive
in a few problems. This set is very restricted to equations with homogeneous coefficients.
Furthermore, the method requires the knowledge of some analytical tools, such
representations formulas and fundamental solutions.
\par
An efficient implementation requires the explicit parametrization of the domain, and 
the exact computation of the speed and the curvature of the curve, and of singularities
often contained in integral kernels (at least for equations constructed from BVPs).
This appears clearer for the computation of the volume potential
\begin{equation}
 v(z_0)\coloneqq-\int_\Omega\Phi(z_0, z)f(z)\,dz\quad z_0 \in \overline{\Omega},
\end{equation}
to solve the non homogeneous Laplace equation $\Delta u = f$. The exact integration
of the singularity, with the help of an explicit expression, can avoid to 
define an adaptive quadrature mesh, near the singularity, for any point $z_0$.
This would make the method impracticable.
\par
The advantage is the fast exponential convergence in $L^\infty$ which gives very good results
for a very small number of nodes, for very simple equations.
\clearpage

% \appendix
% \section{Potential Theory}
% \backmatter
%\nocite{*}
\printbibliography % biber
% \bibliographystyle{plain} % bibtex
% \bibliography{sources} % bibtex
\end{document}

\colorbox{Orchid}{

% \framebox[0.5\textwidth][l]{
 \parbox{0.5\textwidth}{
  in evidenza:
  \begin{description}
         \item[parola chiave 1]: ...sssss;
         \item[parola chiave 2]: ....
  \end{description}

 }
%  }
}


%%%%%%%%%%%%%%%%%%%%%%%%%%%%%%%%%%%%
\begin{alignat}{2}[left=\empheqlbrace]
 & u_t = H(x,t,Du) & \quad&\text{in }\mathbb{R}^n × (0,T) \\[\medskipamount]
  & u(x,0)=u_0(x) & &\text{in } \mathbb{R}^n
\end{alignat}
\vskip 1cm

\begin{subequations}
\begin{alignat}{2}[left=\empheqlbrace]
 & u_t = H(x,t,Du) &\quad & \text{in }\mathbb{R}^n × (0,T) \\[\medskipamount]
 & u(x,0)=u_0(x) & & \text{in } \mathbb{R}^n
\end{alignat}
\end{subequations}
%%%%%%%%%%%%%%%%%%%%%%%%%%%%%%%%%%%%%
\usepackage{cases}
\begin{numcases}{f(x)=}
   1 & $x\geq0$ \label{positive}
   \\
   0 & $x<0$ \label{negative}
\end{numcases}

See the second case \ref{negative} or the first \ref{positive}
% first part is ALREADY MATH MODE
\begin{subnumcases}{f(x)=}
   1 & $x\geq0$ \label{positive-subnum}
   \\
   0 & $x<0$ \label{negative-subnum}
\end{subnumcases}
%%%%%%%%%%%%%%%%%%%%%%%%%%%%%%%%%%%%%%%%
\begin{center}
\begin{tikzpicture}
\draw  plot[smooth, tension=0.7] coordinates {(-3.5,0.5) (-3,2.5) (-1,3.5) (1.5,3) (5,2.5) (5,0.5) (2.5,-2) (-3,-2) (-3.5,0.5)};
\end{tikzpicture}
\end{center}
%%%%%%%%%%%%%%%%%%%%%%%%%%%%%%%%%%%%%%%%%
\begin{tikzpicture}
% \draw [help lines] (-4, -1) grid (4, 5);
\draw [show curve controls]
  (-3, 4) .. controls ++(135:-1) and ++(135:1) .. (0, 4); 
% \draw [show curve controls] (-1, -1) 
%   .. controls ++(165:-1) and ++(270: 1) .. ( 1.5, 1)
  .. controls ++(165:-1) and ++(165:-1) .. ( 0, 1)
  .. controls ++(165: 1) and ++(90: 1) .. (-2, 1)
  .. controls ++(90:-1) and ++(165: 1) .. ( -1, -1);
\end{tikzpicture}
%%%%%%%%%%%%%%%%%%%%%%%%%%%%%%%%%%%%%%%%%%
\tikzstyle{mybox} = [draw=red, fill=blue!20, very thick,
    rectangle, rounded corners, inner sep=10pt, inner ysep=20pt]
\tikzstyle{fancytitle} =[fill=red, text=white]

\begin{tikzpicture}
\node [mybox] (box){%
    \begin{minipage}{0.50\textwidth}
        To calculate the horizontal position the kinematic differential
        equations are needed:
        \begin{align}
            \dot{n} &= u\cos\psi -v\sin\psi \\
            \dot{e} &= u\sin\psi + v\cos\psi
        \end{align}
        For small angles the following approximation can be used:
        \begin{align}
            \dot{n} &= u -v\delta_\psi \\
            \dot{e} &= u\delta_\psi + v
        \end{align}
    \end{minipage}
};
\node[fancytitle, right=10pt] at (box.north west) {A fancy title};
\node[fancytitle, rounded corners] at (box.east) {$\clubsuit$};
\end{tikzpicture}%
%%%%%%%%%%%%%%%%%%%%%%%%%%%%%%%%%%%%%%%%%%
\centering
\includegraphics[width=\textwidth]{fig/prova}
[-1.91529691 -1.97335725 -1.98705091 -1.99249668 -1.99415003 -1.99627237
 -2.00320322 -1.99105893 -2.01682065 -1.97808453 -1.9993953  -1.998383
 -1.99918649 -2.000984   -2.00986185 -1.9862331  -1.99954409 -1.99949211]
[-1.91529691 -1.97335725 -1.98705091 -1.99249668 -1.99415003 -1.99627237
 -2.00320322 -1.99105893 -2.01682065 -1.97808453 -1.9993953  -1.998383
 -1.99918649 -2.000984   -2.00986185 -1.9862331  -1.99954409 -1.99949211]
out/mesh_one_ellipse_2_1_50
[-1.92146356 -1.97336453 -1.98705427 -1.99249722 -1.9941497  -1.99627168
 -2.00320611 -1.99105791 -2.01682316 -1.97808103 -1.99939492 -1.99774228
 -1.99987823 -2.00098443 -2.00986266 -1.98619739 -1.9995809  -1.99948914]
[-1.03384921 -2.80675339 -1.65444285 -1.95018959 -1.95664611 -1.96981612
 -1.97923311 -1.97500048 -2.00380637 -1.96838774 -1.98940744 -1.98695238
 -1.99444467 -1.99887955 -2.00498359 -1.98170047 -1.99333241 -1.73884316]
out/mesh_one_ellipse_2_1_50
[-1.92328738 -1.97336476 -1.98705427 -1.99249722 -1.9941497  -1.99627168
 -2.00320611 -1.99105791 -2.01682316 -1.97808103 -1.99939492 -1.99774228
 -1.99987823 -2.00098443 -2.00986266 -1.98619739 -1.9995809  -1.99948914]
[-0.28484034 -2.50033809 -1.72562427 -1.94575414 -1.95680608 -1.96981341
 -1.97923304 -1.97500048 -2.00380636 -1.96838774 -1.98940744 -1.98695238
 -1.99444467 -1.99887955 -2.00498359 -1.98170047 -1.99333241 -1.73884316]
out/mesh_one_ellipse_2_1_50
used gn 10
out/mesh_one_ellipse_2_1_50
[-1.92328738 -1.97336476 -1.98705427 -1.99249722 -1.9941497  -1.99627168
 -2.00320611 -1.99105791 -2.01682316 -1.97808103 -1.99939492 -1.99774228
 -1.99987823 -2.00098443 -2.00986266 -1.98619739 -1.9995809  -1.99948914]
